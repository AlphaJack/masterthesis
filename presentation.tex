% !TEX TS-program = lualatex
% !TEX encoding = utf-8
% !TEX spellcheck = en-US
% !BIB TS-program = biber

% ┌───────────────────────────────────────────────────────────────┐
% │ Contents of presentation.tex                                  │
% ├───────────────────────────────────────────────────────────────┘
% │
% ├──┐PREAMBLE
% │  ├── CLASS OPTIONS
% │  ├── LAYOUT OPTIONS
% │  └── VARIABLES
% ├──┐BODY
% │  ├── TITLE
% │  ├── OUTLINE
% │  ├── CONTENT
% │  ├── REFERENCES
% │  └── CLOSING
% │
% └───────────────────────────────────────────────────────────────

% ################################################################ PREAMBLE

% standard suitable for digital preservation
% can increase compilation time by 3.5x, uncomment only for final version
%\DocumentMetadata{
% pdfversion=2.0,
% testphase=phase-II,
% pdfstandard=A-4
%}

% ################################ CLASS OPTIONS

\documentclass[
 11pt,
 % uncomment to print white pages with only slide notes, remove semi-transparency slides from {show only notes} option
 %handout,
 % comment out for 4:3
 aspectratio=169,
 % option to maintain a consistent text width in case of multiple columns
 onlytextwidth,
 % align slides content to top by default
 t
]{beamer}

% ################################ LAYOUT OPTIONS

% uncomment to print white pages with only slide notes 
%\setbeameroption{show only notes}

% use custom theme and load required packages
% other nice themes: Berlin Boadilla CambridgeUS Ilmenau Madrid Szeged
\usetheme{masterthesis}

% uncomment to make the presentation full-screen when the file is opened
%\hypersetup{pdfpagemode=FullScreen}

% ################################ VARIABLES

% load variables from external file
% must be loaded after using the beamer theme (new packages are loaded)
% !TEX TS-program = lualatex
% !TEX encoding = utf-8
% !TEX root = thesis.tex
% !TEX spellcheck = en-US
% !BIB TS-program = biber

% ┌───────────────────────────────────────────────────────────────┐
% │ CONTENTS OF variables.tex                                     │
% ├───────────────────────────────────────────────────────────────┘
% │
% ├── STUDENT
% ├── COLORS
% ├── FILES
% ├── METADATA
% ├── FONTS
% │
% └───────────────────────────────────────────────────────────────

% ################################################################ STUDENT

\def\universityName{School of Hard Knocks}
\def\departmentName{Department of Economics}
\def\degreeName{Master of Economic Markets}
\def\thesisTitle{Modeling Volatility and Equilibrium Dynamics in Buyer-Seller Interactions}
\def\thesisSubject{Buyer-Seller interactions}
\def\thesisKeywords{buyer-seller interactions, economic dynamics, market modeling, market equilibrium, volatility analysis}
\def\authorName{Bertram Gilfoyle}
\def\authorMatriculation{4206942}
\def\supervisorName{Prof. Peter Gregory}
\def\academicYear{2022--2023}
\def\graduationYear{2023}
\def\graduationDate{November 01, \graduationYear}

% ################################################################ COLORS

% silence commands if xcolor is not loaded
\providecommand{\definecolor}[3]{}
\providecommand{\colorlet}[2]{}

% primary color
% cover: used for background
% document: used for title, chapter number and drop cap, links
% presentation: used for title, section and thank you page, alerted text

% secondary color
% document: used for headings, page numbers, footer
% presentation: used for blocks

% define primary and secondary colors
% unipd
\definecolor{PrimaryColor}{HTML}{9B0014}% dark red, pantone 1807
\definecolor{SecondaryColor}{HTML}{484F59}% dark grey, pantone 432
% esc
%\definecolor{PrimaryColor}{HTML}{ED7F2A}% orange
%\definecolor{SecondaryColor}{HTML}{585757}% dark grey

% apply custom definitions
\colorlet{LinkColor}{PrimaryColor}
\colorlet{NumberColor}{PrimaryColor}
\colorlet{TitleColor}{SecondaryColor}
\colorlet{HeaderColor}{SecondaryColor}
\colorlet{FooterColor}{SecondaryColor}
\colorlet{FootnoteColor}{SecondaryColor}
\colorlet{PresentationPrimaryColor}{PrimaryColor}
\colorlet{PresentationSecondaryColor}{SecondaryColor}

% ################################################################ FILES

% loads images even if in draft mode
% https://tex.stackexchange.com/a/162388/213962
\setkeys{Gin}{draft=false}

% no need to specify "pictures/" folder
\graphicspath{
 {./pictures}
 {../}% book cover
}

% no need to specify "chapters/" folder
%\makeatletter
%\def\input@path{
% %{.}
% {chapters}
% {pages}
% {output}
%}
%\makeatother

% logos
\def\departmentLogoFg{dsea-red}
\def\universityLogoFg{unipd-red-circle}
\def\universityLogoBg{unipd-black-circle}
\def\departmentLogoFgCover{dsea-white}
\def\universityLogoFgCover{unipd-white-circle}
\def\universityLogoBgCover{\universityLogoBg}
\def\universityLogoFgPresentation{unipd-white-full}
\def\departmentLogoFgPresentation{\departmentLogoFgCover}
\def\universityLogoBgPresentation{\universityLogoBgCover}

% ################################################################ METADATA

% silence command if hyperref is not loaded
\providecommand{\hypersetup}[1]{}

\hypersetup{
 pdftitle={\thesisTitle},
 pdfsubject={\thesisSubject},
 pdfkeywords={\thesisKeywords},
 pdfauthor={\authorName},
 pdfcreator={LaTeX},% default: LaTeX with hyperref
 pdfproducer={LuaLaTeX}% default: LaTeX
}

\author{\authorName}
\title{\thesisTitle}

% ################################################################ FONTS

% fonts are loaded on a document basis  (document, cover, presentation) and not here,
% to avoid unnecessary font loading that would slow down the compilation process
% here we write common font features, since the same fonts are loaded in different documents

% add to each document the required commands:

%\setmainfont{Minion 3}
%\setmonofont{Fira Code}
%\setmathfont{Libertinus Math}
%\setmathrm{Libertinus Math}
%\renewfontfamily\TitleNumberFont{Tex Gyre Pagella}
%\renewfontfamily\TitleFont{Minion 3}
%\renewfontfamily\HeaderFont{Minion 3}
%\renewfontfamily\InitialsFont{Minion 3}



% set variables
\title{\thesisTitle}
\author{\authorName}

% add references
%\addbibresource{references.bib}

% select font
%\setsansfont{Raleway}

% ################################################################ BODY

\begin{document}

% only include specific frames to speedup compilation time
%\includeonlyframes{example1,example3}

% ################################ TITLE

\addTitleSlide

% ################################ OUTLINE

\addOutlineSlide

% ################################ CONTENT

\section{Text and lists}

\begin{frame}{Title of this slide}
 This is normal text. This is \alert{important text}.
 \begin{block}{Definition}
  It is what it is.
 \end{block}
\end{frame}

\begin{frame}{Unnumbered progressive list}
 \begin{itemize}
  \item<1-> First item
  \item<2-> Second item
  \item<3-> Third item
 \end{itemize}
\end{frame}

\begin{frame}{Unnumbered focused list}
 \begin{itemize}
  \item<1> First item
  \item<2> Second item
  \item<3> Third item
 \end{itemize}
\end{frame}

\begin{frame}{Numbered list}
 \begin{enumerate}
  \item First level
  \begin{enumerate}
   \item Second level
  \end{enumerate}
  \item First level
 \end{enumerate}
\end{frame}

\begin{frame}{Title of my frame}
 This text will appear in the presentation. \note{Explain why this note text will not appear}
\end{frame}

%\begin{frame}{Title of this slide}
% \textcite{RUSSELL2020} are cool guys.
% This is not normal text \autocite{RUSSELL2020}.
%\end{frame}

\section{Columns}

\begin{frame}{Two columns}
 \begin{outerColumns}
  \innerColumnsTwo{
   Text of the first column
  }{
   Text of the second column
  } 
 \end{outerColumns}
\end{frame}

\begin{frame}{Three columns}
 \begin{outerColumns}
  \innerColumnsThree{
   Text of the first column
  }{
   Text of the second column
  }{
   Text of the third column
  } 
 \end{outerColumns}
\end{frame}

\begin{frame}{Four columns}
 \begin{outerColumns}
  \innerColumnsFour{
   Text of the first column
  }{
   Text of the second column
  }{
   Text of the third column
  }{
   Text of the four column
  } 
 \end{outerColumns}
\end{frame}

\section{Images and Tables}

\begin{frame}[c]{Vertically-centered, all-frame image}
 \begin{outerImage}
  \innerImageOne{decoration}{A standard decoration}
 \end{outerImage}
\end{frame}

\begin{frame}{Image Two}
 \begin{outerColumns}
  \innerColumnsTwo{
   \begin{outerImage}
    \innerImageOne{decoration}{A standard decoration}
   \end{outerImage}
  }{
   Other text
  } 
 \end{outerColumns}
\end{frame}

\begin{frame}{Table}
 \begin{outerTable}
  \caption{caption}
  \begin{innerTable}{lrr}
   Treatments  & Response 1 & Response 2\\
   Treatment 1 & 0.0003262 & 0.562 \\
   Treatment 2 & 0.0015681 & 0.910 \\
   Treatment 3 & 0.0009271 & 0.296 \\
  \end{innerTable}
 \end{outerTable}
\end{frame}

\addPlaceholderSlide{ADD CONCLUSIONS}


% ################################ REFERENCES

%\addReferencesSlide

% ################################ CLOSING

\addThankYouSlide

\end{document}
